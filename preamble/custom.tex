% Include any further packages/commands/macros/etc in this file.
% Below are some examples, which are used in the demo document.


% for filler text
\usepackage{lipsum}


% tikz (all tikz-related commands should be excluded from the HTML version)
\iflatexml\else
\usetikzlibrary{arrows}
\fi

% Code listings settings
\lstset{basicstyle={\small\ttfamily},language=[LaTeX]TeX,%
  texcsstyle=*{\color{blue}\bfseries},%
  moretexcs={ifcsname,excludecomment},%
  stringstyle={\color{red}},%
  frame=none,%
  xleftmargin=1em,xrightmargin=1em}


%% Custom environments

% FAQ answers
\newenvironment{ans}
{\begin{itemize}\item[]\textbf{A:}}{\end{itemize}}

% Collapsible code snippet
\iflatexml
    \newenvironment{snippet}[1][Code snippet]
        {\<details style="text-align: left; width: 100\%" open="">\<summary>\textbf{#1}\</summary>\begin{lstlisting}}
        {\end{lstlisting}\</details>}
\else
    \newenvironment{snippet}[1][Code snippet]
        {\begin{lstlisting}[caption=#1]}
        {\end{lstlisting}}
\fi

% collapsible solution
\begin{comment}
\iflatexml  
    \renewenvironment{solution} 
    {\<details style="text-align: left; width: 100\%" closed="">  
    \<summary>  \textbf{Solution}  \</summary>  \\} 
    {\</details> } 
\else 
    \usepackage{version}  
    % \excludeversion{solution}  
    \includeversion{solution}  
\fi 
\end{comment}

%% Macros

% arg min and arg max
\DeclareMathOperator*{\argmax}{arg\,max}
\DeclareMathOperator*{\argmin}{arg\,min}

% sign
\DeclareMathOperator{\sign}{sign}

% real and complex numbers
\newcommand{\R}{\mathbb{R}}
\newcommand{\C}{\mathbb{C}}

% norm
\newcommand{\norm}[1]{\left\Vert #1 \right\Vert}

% bold for vectors and matrices
\newcommand{\vect}[1]{\mathbf{#1}}
\newcommand{\mat}[1]{\mathbf{#1}}

% \dd for straight differential (use as \dd x or \dd{x})
\newcommand{\dd}[1]{\mathop{\mathrm{d}#1}}


\usepackage{float}

\floatstyle{ruled}
\newfloat{program}{thp}{lop}
\floatname{program}{Program}
